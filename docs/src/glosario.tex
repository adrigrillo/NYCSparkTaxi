% Acrónimos

% TODO: Añadir aquí los acrónimos
% Ejemplo de acrónimo
\newacronym{DEBS}{DEBS}{Distributed Event-Based Systems}
\newacronym{AEPD}{AEPD}{Agencia Española de Protección de Datos}
\newacronym{FOIL}{FOIL}{Freedom of Information Law}
\newacronym{WDC}{WDC}{World Data Center}
\newacronym{ONU}{ONU}{Organización de las Naciones Unidas}
\newacronym{API}{API}{Interfaz de Programación de Aplicaciones}
\newacronym{AC}{A.C}{Antes de Cristo}
\newacronym{CPU}{CPU}{Unidad Central de Procesamiento}
\newacronym{RAM}{RAM}{Memoria de Acceso Aleatorio}
\newacronym{CSV}{CSV}{Comma-separated values}
\newacronym{GB}{GB}{Gigabyte}
\newacronym{NFS}{NFS}{Network File System o Sistema de archivos de red}
\newacronym{HDFS}{HDFS}{Hadoop Distributed File System}
\newacronym{SO}{S.O.}{Sistema Operativo}
\newacronym{SSH}{SSH}{Secure Shell}
\newacronym{USB}{USB}{Universal Serial Bus}
\newacronym{BIOS}{BIOS}{Basic Input/Output System}
\newacronym{JVM}{JVM}{Máquina Virtual de Java}
\newacronym{PIP}{PIP}{Instalador de Paquetes de Python}
\newacronym{SQL}{SQL}{Structured Query Language}
\newacronym{AWS}{AWS}{Amazon Web Services}
\newacronym{ITU}{ITU}{International Telecommunication Union}
\newacronym{BDaaS}{BDaaS}{Big Data as a Service}
\newacronym{RDD}{RDD}{Resilient Distributed Dataset}
\newacronym{DAG}{DAG}{Direct Acyclic Graph}






% Glosario

% TODO: Añadir aquí las definiciones del glosario
% Ejemplo de glosario
\newglossaryentry{derOlvido}{name={Derecho al Olvido},description={Hace referencia al derecho a impedir la difusión de información personal a través de Internet cuando la información es obsoleta o ya no tiene relevancia ni interés público, aunque la publicación original sea legítima}}
\newglossaryentry{FOILPET}{name={Freedom of Information Law},description={Ley que permite a un individuo tener el derecho de ver y/o obtener ciertos registros realizado por el gobierno}}
\newglossaryentry{framework}{name={Framework},description={Conjunto estandarizado de conceptos, prácticas y criterios para enfocar un tipo de problemática particular que sirve como referencia, para enfrentar y resolver nuevos problemas de índole similar.}}
\newglossaryentry{USBLive}{name={USB Live},description={Instalación de una distribución Linux en un pendrive para que esta puede ser ejecutada sin instalarse en el disco duro del ordenador al que se haya enchufado la memoria USB. Permitiendo su ejecución de esta forma o la instalación del sistema operativo en el ordenador.}}
