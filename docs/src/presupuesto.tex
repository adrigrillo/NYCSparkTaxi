\chapter{Impacto socioeconómico y presupuesto del proyecto \label{sec:presupuesto}}
En este apartado se va a proceder a realizar un análisis de lo que supondría la aplicación del \textit{big data} en el caso del sector del taxi y, posteriormente, se elaborará el presupuesto del proyecto.

Con respecto al presupuesto, se van a analizar los costes de realización que han supuesto la realización de este Trabajo de Fin de Grado. En este, se tendrán en cuenta los medios empleados durante el desarrollo del proyecto para la elaboración de un presupuesto.

\section{Impacto socieconómico}
Durante este documento se ha comentado en varias ocasiones las bondades del \textit{big data} y el positivo impacto que este produce y producirá en la sociedad. En este apartado, vamos a centrarnos en el tema principal de este proyecto, el estudio de los viajes de los taxis, para analizar el impacto la inclusión de estas tecnologías tendrían en el sector.

En este aspecto, ciudades como Nueva York \cite{taxiTrips} o Estocolmo \cite{impacto1} llevan recogiendo datos de los recorridos de taxis para poder realizar análisis sobre estos datos desde hace varios años.

Los posibles usos de estos datos son muy amplios, yendo desde la mejora del tráfico en las ciudades a la de los beneficios del taxista, como en el caso de la segunda consulta. También existes proyectos más enfocados al usuario, como OpenStreetCab \cite{openstreetcab}, que utiliza los datos abiertos de los servicios de taxis de Nueva York y Londres para ofrecer los proveedores más baratos en la situación deseada.

Por tanto, aunque las aplicaciones \textit{big data} implementadas en la realidad son escasas actualmente, existen proyectos de futuras implementaciones que tendrán un impacto positivo tanto en la calidad del servicio como económicamente para todos los participantes, tanto taxistas como para usuarios o para la administración.

Un aplicación que se está estudiando, es la unión de esta información junto con otros datos abiertos de las ciudades, como la meteorología, la agenda de eventos, el estado de otros métodos de transporte y de las carreteras y ciudades para facilitar a las administraciones el control sobre la ciudad y, así, poder destinar recursos para evitar congestiones.

Por otro lado, la apertura de estos datos, también permite que los individuos interesados puedan hacer estudios sobre los datos donde se obtengan resultados beneficiosos para la sociedad. Un ejemplo de este caso, es el estudio sobre el comportamiento de los taxistas en los aeropuertos de Nueva York \cite{aeropuerto} donde se concluye que es mejor esperar a un nuevo cliente que abandonar el aeropuerto.

Por último, un claro ejemplo de éxito de la aplicación del \textit{big data} lo podemos ver en compañias como Uber \cite{uber} o Lyft, que usan análisis en tiempo real para ajustar los precios de los trayectos. También iniciativas como UberPool que, mediante el procesamiento en tiempo real, es capaz de buscar personas que harán trayectos similares a los que hará el usuario y, así, poder compartir taxi.

\section{Presupuesto del proyecto}
\subsection{Medio empleados}
Los medios empleados en el proyecto serán los siguientes:

\begin{itemize}
\item \textbf{Recursos humanos:} Referidos a las personas que han participado en la elaboración del proyecto.
\begin{itemize}
\item \textbf{1 ingeniero junior:} Adrián Rodríguez Grillo.
\item \textbf{1 ingeniero senior:} Pablo Basanta Val.
\end{itemize}

\item \textbf{Medios materiales:} Referidos a los recursos materiales que se utilizan en durante la elaboración del proyecto.
\begin{itemize}
\item 1 ordenador portátil Acer Aspire V5-572PG.
\item 1 ordenador portátil Acer Aspire E5-575G.
\item 1 ordenador de sobremesa Acer VM661-UQ6600C.
\item 1 memoria \gls{USB} Kingstom 16 GB.
\end{itemize}

\clearpage
\item \textbf{Software y licencias:} Referidos a los programas utilizados durante la elaboración del proyecto, tanto con licencia como de código abierto.
\begin{itemize}
\item 3 Ubuntu 16.10.
\item 1 TeXstudio.
\item 1 Microsoft Office 365 Universitario.
\item 1 Apache Spark 2.1.
\item 1 Apache Hadoop 2.7.3.
\item 1 Visual Studio Code.
\end{itemize}

\item \textbf{Medios inmateriales:}
\begin{itemize}
\item Uso del aula de telemática UC3M (40 horas).
\item Conexión a Internet (5 meses).
\end{itemize}
\end{itemize}

\subsection{Presupuesto del trabajo}
Tras especificar los medios utilizados durante la elaboración del proyecto, en este apartado se realizará el desglose de los gastos para elaborar un presupuesto.

\subsubsection{Recursos humanos}
En la elaboración de este proyecto participan dos personas, con diferente posición en el dentro del mismo, por lo que el salario será diferente para cada caso. Se toman como sueldos brutos los siguientes:

\begin{itemize}
\item \textbf{Sueldo ingeniero junior / hora:} 15 euros.
\item \textbf{Sueldo ingeniero senior / hora:} 30 euros.
\end{itemize}

En la tabla \ref{tab:horasRec} se hace un desglose de las horas de dedicación por tarea de los ingenieros que han trabajado en el proyecto y se calculan las horas totales de cada uno. Tras esto, en la tabla \ref{tab:sueldo}, se calcula el coste del trabajo realizado por cada trabajador y el total de ambos sueldos.

\begin{table}[htp!]
\centering
\caption{Desglose de horas de dedicación de los ingenieros implicados en el proyecto}
\label{tab:horasRec}
\begin{tabular}{|l|r|r|r|}
\hline
\multicolumn{1}{|c|}{\textbf{FASE DEL DESAROLLO}}                                                        & \multicolumn{1}{c|}{\textbf{DIAS}} & \multicolumn{1}{c|}{\textbf{\begin{tabular}[c]{@{}c@{}}HORAS \\ INGENIERO \\ JUNIOR\end{tabular}}} & \multicolumn{1}{c|}{\textbf{\begin{tabular}[c]{@{}c@{}}HORAS \\ INGENIERO \\ SENIOR\end{tabular}}} \\ \hline
\begin{tabular}[c]{@{}l@{}}Entendimiento del problema e\\ identificación de las necesidades\end{tabular} & 7                                  & 30                                                                                                 & 5                                                                                                  \\ \hline
Estudio de las tecnologías a utilizar                                                                    & 13                                 & 100                                                                                                & 2                                                                                                  \\ \hline
Diseño del sistema                                                                                       & 15                                 & 80                                                                                                 & 10                                                                                                 \\ \hline
\begin{tabular}[c]{@{}l@{}}Implementación del sistema \\ en el entorno doméstico\end{tabular}            & 17                                 & 90                                                                                                 & 15                                                                                                 \\ \hline
\begin{tabular}[c]{@{}l@{}}Ejecución de pruebas y evaluación \\ del entorno doméstico\end{tabular}       & 10                                 & 40                                                                                                 & 0                                                                                                  \\ \hline
\begin{tabular}[c]{@{}l@{}}Implementación del sistema \\ en el entorno universitario\end{tabular}        & 5                                  & 20                                                                                                 & 15                                                                                                 \\ \hline
\begin{tabular}[c]{@{}l@{}}Ejecución de pruebas y evaluación \\ del entorno universitario\end{tabular}   & 7                                  & 30                                                                                                 & 5                                                                                                  \\ \hline
Análisis de resultados                                                                                   & 9                                  & 40                                                                                                 & 5                                                                                                  \\ \hline
Elaboración de la memoria                                                                                & 15                                 & 60                                                                                                 & 10                                                                                                 \\ \hline
\multicolumn{1}{|r|}{\textbf{TOTAL}}                                                                     & 98                                 & 490                                                                                                & 67                                                                                                 \\ \hline
\end{tabular}
\end{table}

\begin{table}[htp!]
\centering
\caption{Coste total de los recursos humanos}
\label{tab:sueldo}
\begin{tabular}{|l|l|l|l|l|}
\hline
\multicolumn{1}{|c|}{\textbf{CONCEPTO}} & \multicolumn{1}{c|}{\textbf{CANTIDAD}} & \multicolumn{1}{c|}{\textbf{\begin{tabular}[c]{@{}c@{}}HORAS \\ TRABAJADAS\end{tabular}}} & \multicolumn{1}{c|}{\textbf{\begin{tabular}[c]{@{}c@{}}COSTE POR \\ HORA (€)\end{tabular}}} & \multicolumn{1}{c|}{\textbf{\begin{tabular}[c]{@{}c@{}}COSTE \\ TOTAL (€)\end{tabular}}} \\ \hline
Ingeniero junior                        & 1                                      & 490                                                                                       & 15                                                                                          & 7350                                                                                     \\ \hline
Ingeniero senior                        & 1                                      & 67                                                                                        & 30                                                                                          & 2010                                                                                     \\ \hline
\multicolumn{4}{|r|}{\textbf{TOTAL}}                                                                                                                                                                                                                                       & 9360                                                                                     \\ \hline
\end{tabular}
\end{table}

\clearpage
\subsubsection{Medios materiales}
Para este apartado, los valores de la tasa de depreciación se han obtenido de la página de la Agencia Tributaria Española \cite{depreciacion}.

\begin{table}[htp!]
\centering
\caption{Costes de los medios materiales}
\label{costeMediosMateriales}
\begin{tabular}{|l|l|l|l|l|l|}
\hline
\multicolumn{1}{|c|}{\textbf{CONCEPTO}}                         & \multicolumn{1}{c|}{\textbf{\begin{tabular}[c]{@{}c@{}}CAN-\\ TIDAD\end{tabular}}} & \multicolumn{1}{c|}{\textbf{\begin{tabular}[c]{@{}c@{}}PRECIO \\ (€)\end{tabular}}} & \multicolumn{1}{c|}{\textbf{\begin{tabular}[c]{@{}c@{}}PERIODO \\ DE USO\\ (MESES)\end{tabular}}} & \multicolumn{1}{c|}{\textbf{\begin{tabular}[c]{@{}c@{}}TASA \\ DE \\ DEPRECIACIÓN\end{tabular}}} & \multicolumn{1}{c|}{\textbf{\begin{tabular}[c]{@{}c@{}}COSTE \\ TOTAL \\ (€)\end{tabular}}} \\ \hline
\begin{tabular}[c]{@{}l@{}}Acer Aspire \\ V5-572PG\end{tabular} & 1                                                                                  & 750                                                                                 & 5                                                                                                 & 25,00 \%                                                                                         & 112,5                                                                                       \\ \hline
\begin{tabular}[c]{@{}l@{}}Acer Aspire \\ E5-575G\end{tabular}  & 1                                                                                  & 550                                                                                 & 5                                                                                                 & 25,00 \%                                                                                         & 82,5                                                                                        \\ \hline
\begin{tabular}[c]{@{}l@{}}Acer \\ VM661-UQ6600C\end{tabular}   & 1                                                                                  & 599                                                                                 & 5                                                                                                 & 25,00 \%                                                                                         & 89,85                                                                                       \\ \hline
\begin{tabular}[c]{@{}l@{}}USB Kingstom \\ 16 GB\end{tabular}   & 1                                                                                  & 12                                                                                  & 5                                                                                                 & 30,00 \%                                                                                         & 1,68                                                                                        \\ \hline
\multicolumn{5}{|r|}{\textbf{TOTAL}}                                                                                                                                                                                                                                                                                                                                                                                                              & 286,53                                                                                      \\ \hline
\end{tabular}
\end{table}

\subsubsection{Software y licencias}
Para este apartado, los valores de la tasa de depreciación se han obtenido de la página de la Agencia Tributaria Española \cite{depreciacion}.
\begin{table}[htp!]
\centering
\caption{Coste del software y las licencias}
\label{software}
\begin{tabular}{|l|l|l|l|l|l|}
\hline
\multicolumn{1}{|c|}{\textbf{CONCEPTO}}                                      & \multicolumn{1}{c|}{\textbf{\begin{tabular}[c]{@{}c@{}}CAN-\\ TIDAD\end{tabular}}} & \multicolumn{1}{c|}{\textbf{\begin{tabular}[c]{@{}c@{}}PRECIO\\ (€)\end{tabular}}} & \multicolumn{1}{c|}{\textbf{\begin{tabular}[c]{@{}c@{}}PERIODO \\ DE\\ USO\\ (MESES)\end{tabular}}} & \multicolumn{1}{c|}{\textbf{AMORTIZACIÓN}} & \multicolumn{1}{c|}{\textbf{\begin{tabular}[c]{@{}c@{}}COSTE\\ TOTAL \\ (€)\end{tabular}}} \\ \hline
Ubuntu 16.10                                                                 & 3                                                                                  & 0                                                                                  & 5                                                                                                   & 33,00 \%                                   & 0                                                                                          \\ \hline
TeXstudio                                                                    & 1                                                                                  & 0                                                                                  & 5                                                                                                   & 33,00 \%                                   & 0                                                                                          \\ \hline
\begin{tabular}[c]{@{}l@{}}Microsoft Office\\ 365 Universitario\end{tabular} & 1                                                                                  & 0                                                                                  & 5                                                                                                   & 33,00 \%                                   & 0                                                                                          \\ \hline
\begin{tabular}[c]{@{}l@{}}Apache Spark \\ 2.1\end{tabular}                  & 1                                                                                  & 0                                                                                  & 5                                                                                                   & 33,00 \%                                   & 0                                                                                          \\ \hline
\begin{tabular}[c]{@{}l@{}}Apache Hadoop\\ 2.7.3\end{tabular}                & 1                                                                                  & 0                                                                                  & 5                                                                                                   & 33,00 \%                                   & 0                                                                                          \\ \hline
\begin{tabular}[c]{@{}l@{}}Visual Studio\\ Code\end{tabular}                 & 1                                                                                  & 0                                                                                  & 5                                                                                                   & 33,00 \%                                   & 0                                                                                          \\ \hline
\multicolumn{5}{|r|}{\textbf{TOTAL}}                                                                                                                                                                                                                                                                                                                                                                      & 0                                                                                          \\ \hline
\end{tabular}
\end{table}

\clearpage
\subsubsection{Medios inmateriales}
\begin{table}[htp!]
\centering
\caption{Costes de los medios inmateriales}
\label{inmateriales}
\begin{tabular}{|l|l|l|l|l|}
\hline
\multicolumn{1}{|c|}{\textbf{CONCEPTO}} & \multicolumn{1}{c|}{\textbf{CANTIDAD}} & \multicolumn{1}{c|}{\textbf{\begin{tabular}[c]{@{}c@{}}TIEMPO \\ DE\\ USO\end{tabular}}} & \multicolumn{1}{c|}{\textbf{\begin{tabular}[c]{@{}c@{}}COSTE\\ (€)\end{tabular}}} & \multicolumn{1}{c|}{\textbf{\begin{tabular}[c]{@{}c@{}}COSTE\\ TOTAL\\ (€)\end{tabular}}} \\ \hline
Aula telemática                         & 1                                      & 40 horas                                                                                 & 0                                                                                 & 0                                                                                         \\ \hline
Conexión a Internet                     & 1                                      & 5 meses                                                                                  & 45 €/mes                                                                          & 225                                                                                       \\ \hline
\multicolumn{4}{|r|}{\textbf{TOTAL}}                                                                                                                                                                                                                            & 225                                                                                       \\ \hline
\end{tabular}
\end{table}

\subsubsection{Coste total}
\begin{table}[htp!]
\centering
\caption{Coste total del proyecto}
\label{costeTotal}
\begin{tabular}{|l|l|}
\hline
\multicolumn{1}{|c|}{\textbf{CONCEPTO}} & \multicolumn{1}{c|}{\textbf{COSTE (€)}} \\ \hline
Recursos humanos                        & 9360                                    \\ \hline
Medios materiales                       & 286,53                                  \\ \hline
Software y licencias                    & 0                                       \\ \hline
Otros medios                            & 225                                     \\ \hline
\multicolumn{1}{|r|}{\textbf{TOTAL}}    & 9871,53                                 \\ \hline
\end{tabular}
\end{table}

El coste total del proyecto será la suma de los gastos de diferentes tipos de medios. Este coste quedará fijado en 9871.53€ (Nueve mil ochocientos setenta y un euros con cincuenta y tres céntimos) y este se puede encontrar en la table \ref{costeTotal}.